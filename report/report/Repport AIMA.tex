% ---------------------------------------------------------------
% ---------------------------------------------------------------
% This template was developed for the working paper series of 
% the Interdisciplinary Laboratory of Computational Social Science (iLCSS)
% at the University of Maryland, College Park

% The template was built based on  the PNAS Latex model. 

% Adjustments were made by Tiago Ventura, Ph.D. Candidate in Political Science at UMD, and researcher at the iLCSS.

\documentclass[9pt,twocolumn,twoside]{ilcss}
\usepackage{amsthm}
\usepackage{amsmath}
\usepackage{algorithm}
\usepackage[noend]{algpseudocode}
\usepackage[dvipsnames]{xcolor}

\theoremstyle{definition}
\newtheorem{definition}{Définition}[section]

\theoremstyle{remark}
\newtheorem{remark}{Remarque}[section]

\theoremstyle{definition}
\newtheorem*{goal}{Objectif}

\theoremstyle{definition}
\newtheorem*{hypothesis}{Hypothèse}

\templatetype{ilcssworkingpaper} % Choose template 

\title{TER : PACMAN, de la génération des labyrinthes aux agents autonomes}	

% Use letters for affiliations, numbers to show equal authorship (if applicable) and to indicate the corresponding author
\author[a,b]{Arno Lesage (no. 22202985)}
\author[a,c]{Keryann Razafindrabe (no. 22209878)}
\author[a,c]{Jean-Jacques Viale (no. 22202859)}

\affil[a]{Université Côte d'Azur EUR - DS4H - TER Group F}
\affil[b]{Master 1 Informatique, parcours IA}
\affil[c]{Master 1 Informatique, parcours Informatique}

% Please include corresponding author, author contribution and author declaration information
\authordeclaration{Template \LaTeX{} utilisé : iLCSS Working Paper Template par Ernesto Calvo et Tiago Ventura sous licence Creative Commons CC BY 4.0.}
\equalauthors{A. Lesage : Agents autonomes (6 demi-pages), K. Razafindrane : Génération de labyrinthes (5 demi-pages), J. Viale : Architecture de code (5.5 demi-pages). Abstract, introduction et conclusion rédigés conjointement. À des fins d'uniformisation (mise en page), il est possible que les auteurs aient modifié des parties ne leur étant pas assignées. Les auteurs ont contribué à parts égales au projet.}

% Keywords are not mandatory, but authors are strongly encouraged to provide them. If provided, please include two to five keywords, separated by the pipe symbol, e.g:
\keywords{PACMAN $|$ Génération de labyrinthes $|$ Agents autonomes $|$ Architecture et interactions}

%\begin{abstract}
%    PACMAN est un jeu d'arcade composé d'un labyrinthe et d'agents parcourant celui-ci : des fantômes et PacMan, un agent contrôlable humainement. Le but de PacMan est d'explorer l'entièreté du labyrinthe et celui des fantômes de le capturer le plus rapidement possible. Ce jeu, à la logique simple, présente pourtant des défis complexes incluant entre autres la génération procédurale de labyrinthes à propriétés spécifiques ou le développement d'agents autonomes pour les fantômes et PacMan. Ce rapport mets en évidence plusieurs techniques répondant à ces problématiques aussi bien dans la génération des labyrinthes grâce aux polyominos que dans les agents autonomes avec des approches diversifiées parfois collaborative, le tout en précisant les détails d'implémentation technique nécessaires ou facilitant la production et l'implémentation de simulations.
%\end{abstract}

\begin{abstract}
    PACMAN est un jeu d'arcade composé d'un labyrinthe parcouru par plusieurs agents : PacMan, contrôlé par un joueur humain, et des fantômes autonomes. L'objectif de PacMan est d'explorer l'intégralité du labyrinthe, tandis que les fantômes cherchent à le capturer le plus rapidement possible.
    Bien que reposant sur des règles simples, ce jeu soulève des problématiques complexes, telles que la génération procédurale de labyrinthes présentant des propriétés spécifiques, ainsi que la conception d'agents autonomes capables de prises de décision efficaces.
    Ce rapport explore différentes techniques répondant à ces enjeux, notamment l'utilisation des polyominos pour la génération de labyrinthes et diverses approches pour le développement des comportements autonomes des agents incluant collaborations, heuristiques ou l'exploitation de la structure du labyrinthe. Une implémentation du jeu et les interactions nécessaires à la mise en œuvre de simulations fonctionnelles y sont également présentées.
\end{abstract}

%\begin{abstract}
%Please provide an abstract of no more than 250 words in a single paragraph. Abstracts should explain to the general reader the major contributions of the article. References in the abstract must be cited in full within the abstract itself and cited in the text.
%\end{abstract}

\begin{document}

\maketitle
\thispagestyle{firststyle}
\ifthenelse{\boolean{shortarticle}}{\ifthenelse{\boolean{singlecolumn}}{\abscontentformatted}{\abscontent}}{}

% If your first paragraph (i.e. with the \dropcap) contains a list environment (quote, quotation, theorem, definition, enumerate, itemize...), the line after the list may have some extra indentation. If this is the case, add \parshape=0 to the end of the list environment.



%\section*{Introduction}
    In section~\ref{dataset}, we describe the dataset is usage for this project. Section~\ref{methodology} deals with our methodology to work around this project and the challenges we had to overtake. Section~\ref{results} shows the obtained results. Finally, the two last sections encompasses conclusion, perspectives and authors contributions.

\section{Dataset}\label{dataset}
    Dans cette partie, nous explorons différentes méthodes afin de capturer le PacMan le plus rapidement possible. Pour cela, nous devons formaliser notre objectif.

    \subsection{Dataset}
        Lorem Ipsum Lorem Ipsum Lorem Ipsum Lorem Ipsum
        Lorem Ipsum Lorem Ipsum Lorem Ipsum Lorem Ipsum
        Lorem Ipsum Lorem Ipsum Lorem Ipsum Lorem Ipsum
        Lorem Ipsum Lorem Ipsum Lorem Ipsum Lorem Ipsum
        Lorem Ipsum Lorem Ipsum Lorem Ipsum Lorem Ipsum
        Lorem Ipsum Lorem Ipsum Lorem Ipsum Lorem Ipsum
        Lorem Ipsum Lorem Ipsum Lorem Ipsum Lorem Ipsum
        Lorem Ipsum Lorem Ipsum Lorem Ipsum Lorem Ipsum
        Lorem Ipsum Lorem Ipsum Lorem Ipsum Lorem Ipsum
        Lorem Ipsum Lorem Ipsum Lorem Ipsum Lorem Ipsum
        Lorem Ipsum Lorem Ipsum Lorem Ipsum Lorem Ipsum
        Lorem Ipsum Lorem Ipsum Lorem Ipsum Lorem Ipsum
        Lorem Ipsum Lorem Ipsum Lorem Ipsum Lorem Ipsum
        Lorem Ipsum Lorem Ipsum Lorem Ipsum Lorem Ipsum
        Lorem Ipsum Lorem Ipsum Lorem Ipsum Lorem Ipsum
        Lorem Ipsum Lorem Ipsum Lorem Ipsum Lorem Ipsum
        Lorem Ipsum Lorem Ipsum Lorem Ipsum Lorem Ipsum
        Lorem Ipsum Lorem Ipsum Lorem Ipsum Lorem Ipsum
        Lorem Ipsum Lorem Ipsum Lorem Ipsum Lorem Ipsum
        Lorem Ipsum Lorem Ipsum Lorem Ipsum Lorem Ipsum

\section{Methodology}\label{methodology}
    Lorem Ipsum

\section{Results}\label{results}
    Lorem Ipsum

    \subsection{Empty B}
        Lorem Ipsum

        \subsubsection{Résultats et discussions}
            Lorem Ipsum

            Lorem Ipsum Lorem Ipsum Lorem Ipsum Lorem Ipsum
            Lorem Ipsum Lorem Ipsum Lorem Ipsum Lorem Ipsum
            Lorem Ipsum Lorem Ipsum Lorem Ipsum Lorem Ipsum
            Lorem Ipsum Lorem Ipsum Lorem Ipsum Lorem Ipsum
            Lorem Ipsum Lorem Ipsum Lorem Ipsum Lorem Ipsum
            Lorem Ipsum Lorem Ipsum Lorem Ipsum Lorem Ipsum
            Lorem Ipsum Lorem Ipsum Lorem Ipsum Lorem Ipsum
            Lorem Ipsum Lorem Ipsum Lorem Ipsum Lorem Ipsum
            Lorem Ipsum Lorem Ipsum Lorem Ipsum Lorem Ipsum
            Lorem Ipsum Lorem Ipsum Lorem Ipsum Lorem Ipsum
            Lorem Ipsum Lorem Ipsum Lorem Ipsum Lorem Ipsum
            Lorem Ipsum Lorem Ipsum Lorem Ipsum Lorem Ipsum
            Lorem Ipsum Lorem Ipsum Lorem Ipsum Lorem Ipsum

        %%%%%% ------------------------------------------------------------------- %%%%%



\section*{Conclusion and Perspectives}

In this work, we cite \cite{placeholder}.

\section*{Author's contrinutions}

\acknow{None}
\showacknow{} % Display the acknowledgments section

\bibliography{references}


\end{document}